\documentclass[]{article}

\usepackage[tmargin=1in,bmargin=1in,lmargin=1.25in,rmargin=1.25in]{geometry}
\usepackage{titlesec}
\titleformat*{\section}{\Large\bfseries\sffamily}
\titleformat*{\subsection}{\large\bfseries\sffamily}
\usepackage{amsmath}
\usepackage{algorithm}
\usepackage[noend]{algpseudocode}

\makeatletter
\def\BState{\State\hskip-\ALG@thistlm}
\makeatother

\setlength\parindent{15pt}

\begin{document}

\section*{Transition Path Sampling}
\textbf{Author}: Dylan Goldsborough\\
\textbf{Date}: July 8, 2016

\subsection*{Internship Goals}

The primary goal of this 6 ECTS internship was to produce a tutorial for the \textsc{tipsi} package, developed by Tsjerk Wassenaar \cite{tipsi}. This application of transition path sampling (TPS) was completed by Tsjerk as a PhD, but documentation was never written for the package. To make the package accessible to use for students and members of the computational chemistry group that did not have hands-on experience with the package yet, I were to write a tutorial that introduces \textsc{tipsi} through  one or more examples, and provide some minimal documentation. The documentation itself is not aimed to have full coverage, but sufficient coverage to allow users to run \textsc{tipsi} without having to dive into the source code themselves. The tutorial and documentation is to be available on \textsc{git}.

A secondary goal was to write a few scripts that can help users work with \textsc{tipsi}. The likely application of these scripts was some form of processing of the output. This goal was considered optional, as the need (or lack thereof) for these tools would become apparent in the process of developing the tutorial.

\subsection*{Realization}

The contributions I realized in this internship fall in a few broad categories. Each of these categories I will discuss briefly, to show my methods in developing this product, to discuss what goal this helps fullfill, and to discuss some of the pitfalls and solutions found for these problems. The categories are as follows:
%
\begin{itemize}
\item Bash and Python scripts for processing \textsc{tipsi output}
\item A tutorial with accompanying Python script
\item Minimal documentation on setting up and running \textsc{tipsi}
\end{itemize}
%
I also encountered some miscelleaneous facts that were not widely known in the group, I will discuss those in a final subsection.

\subsubsection*{Bash and Python scripts}

A major challenge in \textsc{tipsi} is the assembly of the transition paths it generates. Applications of TPS, such as \textsc{tipsi}, use an input trajectory that describes the transition from a state \textit{A} to a state \textit{B} of a molecular system, and generates many more trajectories. It does this by a process referred to as 'shooting'. In this process, a time called the 'shooting point' is selected, after which the simulation is continued from that state in a forward or backward direction (with slightly altered velocities). If this 'shooting move' reaches the same state as this part of the 'parent trajectory' did, then the move is accepted, and the move becomes part of a new trajectory. Repeating this process results in multiple trajectories, that consist of frames from a number of different shooting moves. To save space, \textsc{tipsi} only stores the shooting moves, as storing the full trajectories would mean that frames are stored multiple times. Up until this point, no tool had been written to actually assemble the output of \textsc{tipsi} into the full trajectories.

Assembling the trajectories was an unforeseen challenge, due to some of the properties of the molecular dynamics engine that is used: \textsc{gromacs} \cite{gromacs}. This software is only designed for simulations that go forward in time, as it was not build with TPS in mind. \textsc{Tipsi} works around this by using a negative timestep in the shooting moves that go backkward in time. However, the resulting trajectory cannot be processed by \textsc{gromacs} due to the decreasing timestamps. This problem is worsened by the fact that \textsc{tipsi} produces negative timestamps. Negative numbers are used in the \textsc{gromacs} command line parameters as a null-indicator, making it impossible to retrieve frames with a negative timestamp. As a result of this, it is not possible to simply turn the order of the frames in backward shooting moves around to process them.  

I solved this problem by writing a Bash script, which I appended to the \textsc{tipsi} runscript. This way, the Bash script is automatically ran before the data is copied back from the computing node to the directory of the user. The script follows the following outline in its functioning:
%
\begin{algorithm}
\caption{Assembling Transition Paths}\label{euclid}
\begin{algorithmic}[1]
\State $\textit{regex} \gets \text{regex extracting \textit{timestamp} and \textit{framesource}}$
\For{directories with ACCEPTED}
\For{each \textit{line} in dat-file}
\State \textbf{apply} \textit{regex} \textbf{to} \textit{line}
\State \textbf{overwrite} timestamps \textbf{from} \textit{framesource}
\State \textbf{extract} \textit{frame} \textbf{from} \textit{framesource}
\State \textbf{append} \textit{frame} \textbf{to} \textit{output}
\EndFor
\State \textbf{store} data on trajectory \textbf{to} \textit{csv}
\State \textbf{save} \textit{output}
\EndFor
\State \textbf{save} \textit{csv}
\end{algorithmic}
\end{algorithm}
%
This is a highly simplified version of the script, as finding which frame to extract after overwriting the timestamps is not straightforward. We decided to make each complete trajectory start at $t=0$, and naturally use the same timestep as the original trajectory.

I wrote a Python script to analyze the \texttt{csv} that this script outputs. The analysis is primarily meant to give the user a good idea of what the shooting process has looked like, and spot if something went wrong. This script outputs the average length of the transition paths, the number of decorrelated paths, and the ratio between succesfull forward and backward shooting moves. In addition, it draws a plot which visualizes the shooting process using a Python drawing tool. A minor drawback of this visualization method is that this script, unlike all other scripts I wrote, cannot be run on a cluster, as this package does not run when monitor output is not available. This should not be a problem, however, as visualization of the trajectory is also done on a personal computer.

\subsubsection*{Tutorial}

I wrote a tutorial that contains two examples, as well as a brief introduction to the general idea of TPS. The first example is designed to have a low computational time, and be easy to analyze. The seconds example was selecte to be more demanding, but to be an actual example of a rare event. In the repository containing the tutorial, I also included some sample output for students to reference their results with.

For the first example, the molecule alanine dipeptide was chosen. This compound has two distinct states, between which transitions occur. These two states can be described using two dihedral angels, and are visually distinguishable. To give the student the complete picture of what working with \textsc{tipsi} entails, this example lets the student start from scratch. We start with a file describing the structure of alanine dipeptide, and files containing settings for molecular dynamics simulations. We prepare the simulation by placing the molecule in a solvent, in a periodic box. We do two regular simulations with this system: one at room temperature (298K), and one at high temperature (400K). The former is used to describe the stable states, and to see if these states are stable enough. I wrote a Python scripts which produces plots, from which the ranges between which the system is in a stable state can be read for two order parameters. This information is used to set the state definitions in the \textsc{tipsi} parameter file. The high temperature simulation is used to provide an initial transition to \textsc{tipsi}, as a higher temperature can force a transition.

For the second example, a project by two members of the computational chemistry group (Jocelyne Vreede and David Swenson) was used. This project concerned the transition in a DNA basepair from the Watson-Crick configuration to the Hoogsteen configuration. This event is actually rare, as it does not occur in normal molecular dynamics simulations of a length up to microseconds (approximately the upper limit of simulation lengths at this moment). Jocelyne and David have obtained several transitions already by forcing the transition using a metadynamics simulation, and have already defined the states. As such, this example is less work for the student: all that is left is setting up a file that contains the conditions for the simulations \textsc{tipsi} produces, and setting up \textsc{tipsi} itself. With this example, students are shown that \textsc{tipsi} does what it promises. In addition, it shows students what the shooting process is like in a more complex molecular system; in the first example the shooting process is nice and symmetrical, whereas in this example this is not the case due to the increased complexity of the system.

\subsubsection*{Documentation}

For documentation, I wrote three different pieces. Two of these are aimed at students in the biomolecular simulations course, and contain a 'cheat sheet' for Linux and \textsc{gromacs} commands. This was included as I found, while taking this course, that many where unfamiliar with one or both. The third piece is documentation for \textsc{tipsi}, which documents its use and options.

First, I discuss in this documentation what the running parameters of \textsc{tipsi} are. This provides some insight in how to change the number of shooting moves made, for instance, or how to include a \textsc{gromacs} index file. I also discuss how to set up the parameter file, the second part of the input that \textsc{tipsi} requires. In this file, the two states are defined according to some order parameters. Specific atoms are referred to by their atom number, as defined in the \texttt{gro}-file used to set up the simulation.

In this part, I encountered a major indexing issue in \textsc{tipsi}. This issue cannot properly be resolved, but I made sure to clearly warn users in the tutorial, as this has to be kept in mind while writing the parameter file. If not, the TPS process is guaranteed to fail. Declaring groups of atoms on the parameter file is done using numpy-arrays.  Python, as most programming languages, is zero-indexed, meaning that arrays start counting at zero. This means that, when the list of all atoms are stored in Python, the first atom is labeled '0'. \textsc{Gromacs}, however, is one-indexed, meaning that the first atom in \texttt{gro}-files is labeled '1'. To add to the confusion, there is the possibility to read in groups of atoms from index files. In this case the atom numbers are one-indexed as well. As such, to make sure that the atom numbers match the correct atom, 1 has to be subtracted whenever entering atom numbers directly into the parameter file.

Finally, I read through the section of the \textsc{tipsi} source code that does order parameter calculations. I made a comprehensive list of all the possible functions that can be used to define states. I made sure to find the highest level function for each, and found that in some of the existing parameter files functions were used that were not meant to be called directly. For instance, \texttt{pdist} is an often-used order parameter, as it finds the distance in a periodic box. However, it turns out that the function \texttt{dist} can be called with a running parameter to use a periodic box, which will make it utilize the \texttt{pdist} function. In addition to a specifying a periodic box, other running parameters can be added then as well, such as the \texttt{what} parameter, which allows you to use Euclidean or Manhattan distances.

A minor pitfall in the calculator functions I found, is with the radius of gyration. Most of the calculator functions return the same value as the corresponding \textsc{gromacs} alternatives, within a reasonable rounding error. However, the radius of gyration is calculated in a different manner: \textsc{gromacs} uses the atomic mass of each atom to weigh their contribution to the radius of gyration, whereas \textsc{tipsi} does not. As such, this parameter can only be used if \textsc{tipsi} is used to define the states. This is possible, however, as running \textsc{tipsi} without state definitions will simply make it track the values of the order parameters throughout a trajectory. 

\subsubsection*{Miscelleaneous}

While working on this project, I found that \textsc{tipsi} runs the backward shooting moves by using a negative timestep. This was not widely known in the group, as the consensus was that \textsc{tipsi} worked by multiplying the velocities with -1.

\subsection*{Summary}

To summarize, I wrote a tutorial on \textsc{tipsi} to be used by students and researchers in the group that want to get hands-on experience with the package. I included minimal documentation, that ensures that \textsc{tipsi} can be used without forcing the user to read the sourcecode. I also included cheatsheets for students that are less familiar with Linux and \textsc{gromacs}. I finally also wrote a Bash and Python script that assemble the output of \textsc{tipsi} into a form that can be analyzed by researchers.

To conclude the project, I presented an overview of my work to the computational chemistry group on the 8th of July, 2016. In this half-hour presentation, I discussed both examples in the tutorial, showing results, outlined the documentation I wrote, and discussed the Bash and Python scripts I wrote to process \textsc{tipsi} output.

\bibliographystyle{plain}
\bibliography{references}

\end{document}